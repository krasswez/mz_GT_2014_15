\documentclass{scrartcl}


\usepackage{amsmath}	  % required for math in general
\usepackage{amsthm}     % environments for theorems, qed's etc
                        % (loaded after amsmath)
\usepackage{amssymb}	  % doublestroke symbols, other mathematical symbols
\usepackage{dsfont}     % required for double-stroke 1 as characteristic function
\usepackage{array}	    % control of matrices and tables
\usepackage{graphicx}   % images

\usepackage{enumitem}   % more fine-grained control over enumerations
\setdescription{leftmargin=\parindent,labelindent=\parindent}

\usepackage{listings} % code listings
\lstset{basicstyle=\ttfamily\scriptsize}

% \input{diagrams.sty} (no category theory this time)

\usepackage{helvet}   % use (much fresher looking) helvetica for everything
\renewcommand{\familydefault}{\sfdefault}

\usepackage[weather]{ifsym}      % \Lightning symbol
% \usepackage{mathabx}             % \Asterisk causes some conflicts

% forcing the fucking floats to stop fucking floating like a fucking piece of
% shit in an ocean of fucking shit
\renewcommand{\topfraction}{.85}
\renewcommand{\bottomfraction}{.7}
\renewcommand{\textfraction}{.15}
\renewcommand{\floatpagefraction}{.66}
\renewcommand{\dbltopfraction}{.66}

% making all references into hyperlinks
\usepackage[dvipsnames]{xcolor}
\usepackage{hyperref}

\hypersetup{colorlinks=true,linkcolor=MidnightBlue,pdfborderstyle={/W 0}}

\usepackage{anyfontsize}
\usepackage{datetime}

% forall
\let\oldforall\forall
\renewcommand{\forall}{\oldforall\,}

% parentheses
\newcommand{\rPar}[1]{\left(#1\right)} % round parens
\newcommand{\sPar}[1]{\left[#1\right]} % square parens
\newcommand{\cPar}[1]{\left\{#1\right\}} % curved parens 
\newcommand{\aPar}[1]{\left\langle #1 \right\rangle} % angle brackets

% floor and ceiling
\newcommand{\floor}[1]{{\left\lfloor#1\right\rfloor}} % curved parens 
\newcommand{\ceil}[1]{{\left\lceil#1\right\rceil}} % curved parens 

% norms
\newcommand{\abs}[1]{\left\lvert #1\right\rvert}
\newcommand{\norm}[1]{\left\lVert #1\right\rVert}
\newcommand{\scalar}[2]{\left\langle#1,#2\right\rangle}
\newcommand{\cross}{\times}
\DeclareMathOperator{\diam}{diam}
\DeclareMathOperator{\B}{B}

% intervals
\newcommand{\openOpenInterval}[2]{{\left(#1,#2\right)}}
\newcommand{\openClosedInterval}[2]{{\left(#1,#2\right]}}
\newcommand{\closedOpenInterval}[2]{{\left[#1,#2\right)}}
\newcommand{\closedClosedInterval}[2]{{\left[#1,#2\right]}}

% restriction of functions
\newcommand{\restrict}[2]{{\left.#1\right\vert_{#2}}}

% numbers
\newcommand{\Natural}{\mathbb{N}}
\newcommand{\Integer}{\mathbb{Z}}
\newcommand{\Real}{\mathbb{R}}
\newcommand{\Rational}{\mathbb{Q}}
\newcommand{\PositiveReal}{\Real_{>0}}
\newcommand{\NonnegativeReal}{\Real_{\geq0}}
\newcommand{\Complex}{\mathbb{C}}
\renewcommand{\i}{i}
\newcommand{\Quaternion}{\mathbb{H}}
\newcommand{\Boolean}{\mathbb{B}}

% function spaces
\newcommand{\SemiLebesgue}{\mathscr{L}}
\newcommand{\Continuous}{C}
\newcommand{\Lebesgue}{L}
\newcommand{\Sobolev}{H}
\newcommand{\Hilbert}{\mathscr{H}}
\newcommand{\Schwarz}{\mathscr{S}}

% set
\newcommand{\setPredicate}[2]{{\left\{#1\,\left\vert\, #2\right.\right\}}}
\newcommand{\set}[1]{{\left\{#1\right\}}}
\newcommand{\cardinality}[1]{\left\lvert #1 \right\rvert}
\newcommand{\powerset}[1]{\mathfrak{P}\left(#1\right)}
\DeclareMathOperator*{\intersection}{\bigcap}
\DeclareMathOperator*{\union}{\bigcup}
\newcommand{\disjointUnion}{\biguplus}
\renewcommand{\complement}[1]{#1^c}
% \newcommand{\setminus}{\backslash}
\newcommand{\injective}{\hookrightarrow}
\newcommand{\surjective}{\twoheadrightarrow}
%\DeclareMathOperator{\ker}{ker} % already exists... im does not?
\DeclareMathOperator{\im}{im}

% topological operators
\DeclareMathOperator{\Cl}{Cl}
\newcommand{\Closure}[2]{\Cl_{#1}\left(#2\right)}
\DeclareMathOperator{\const}{const}

% span and conv
\DeclareMathOperator*{\conv}{conv}
\DeclareMathOperator*{\linhull}{span}

% matrices
\newcommand{\mat}[2]{\left[\begin{array}{#1}#2\end{array}\right]}
\DeclareMathOperator*{\diag}{diag}

% landau symbols
\newcommand{\LandauO}[1]{\mathcal{O}\left(#1\right)}

% derivatives
\newcommand{\dd}[2]{\frac{\partial #1}{\partial #2}}
\newcommand{\differential}[1]{\boldsymbol{D}_{#1}}

% integrals
\renewcommand{\d}{\quad\mathrm{d}}

% characteristic functions, expected values, variances, covariances
% stochastic stuff
\newcommand{\one}[1]{\mathds{1}_{#1}}
\newcommand{\weakconv}[1]{\overset{#1}{\Longrightarrow}}
\newcommand{\wlim}{\mathop{\mathrm{wlim}}}
\newcommand{\vlim}{\mathop{\mathrm{vlim}}}
\renewcommand{\P}{\mathbb{P}}
\newcommand{\E}{\mathbb{E}}

% lim inf lim sup
% \DeclareMathOperator{\liminf}{lim inf}
% \DeclareMathOperator{\limsup}{lim sup}

% qed etc.
\renewcommand{\qedsymbol}{$\blacksquare$}
\newcommand{\result}{\hfill $\Diamond$}

% lattices
\newcommand{\meet}{\wedge}
\newcommand{\join}{\vee}
\newcommand{\negate}{\neg}

% listings: Scala
\lstdefinelanguage{scala}{
  morekeywords={abstract,case,catch,class,def,%
    do,else,extends,false,final,finally,%
    for,if,implicit,import,match,mixin,%
    new,null,object,override,package,%
    private,protected,requires,return,sealed,%
    super,this,throw,trait,true,try,%
    type,val,var,while,with,yield},
  otherkeywords={=>,<-,<\%,<:,>:,\#,@},
  sensitive=true,
  morecomment=[l]{//},
  morecomment=[n]{/*}{*/},
  morestring=[b]",
  morestring=[b]',
  morestring=[b]"""
}
\lstset{showstringspaces=false}

% making references look a little nices
\let\oldRef\ref
\renewcommand{\ref}[1]{(\oldRef{#1})}

% weird stuff for computer science
\DeclareMathOperator{\arity}{ar}

% cat, category theory
% Bunch of categories
\DeclareMathOperator{\Id}{Id}
\DeclareMathOperator{\Top}{Top}
\DeclareMathOperator{\hTop}{h-Top}
\DeclareMathOperator{\Sets}{Sets}
\DeclareMathOperator{\Rel}{Rel}
\DeclareMathOperator{\FinSets}{FinSets}
\DeclareMathOperator{\Grp}{Grp}
\DeclareMathOperator{\Cat}{Cat}
\DeclareMathOperator{\Grpd}{Grpd}
\newcommand{\cat}[1]{\mathcal{#1}}
\newcommand{\Obj}{\mathrm{Obj}}
\newcommand{\Hom}{\mathrm{Hom}}
\newcommand{\op}{\mathrm{op}}
\newcommand{\nat}{\xrightarrow{\bullet}}
\newcommand{\iso}{\cong}
\newcommand{\dom}{\mathrm{dom}}
\newcommand{\cod}{\mathrm{cod}}
\DeclareMathOperator{\coeq}{Coeq}
\newcommand{\fst}{\mathrm{fst}}
\newcommand{\snd}{\mathrm{snd}}
\DeclareMathOperator{\Aut}{Aut}
\DeclareMathOperator{\End}{End}

% functors frequently used in various contexts
\DeclareMathOperator{\Free}{Free}
\DeclareMathOperator{\Forget}{Forget}

% empty set that is round
\let\emptyset\varnothing

% generated groups
\newcommand{\gen}[1]{\left\langle#1\right\rangle}
\newcommand{\normalSub}{\triangleleft}
\newcommand{\Asterisk}{\mathop{\scalebox{1.5}{\raisebox{-0.2ex}{$\ast$}}}}
\newcommand{\Sym}{\mathrm{Sym}}

% argmax argmin argsup etc.
\DeclareMathOperator{\argsup}{argsup}

% number theoretic operators
\DeclareMathOperator{\lcm}{lcm}

% get rid of the ugly-looking "epsilon"
\renewcommand{\epsilon}{\varepsilon}

% get rid of the empty-looking "angle"
\renewcommand{\angle}{\measuredangle}

\DeclareMathOperator{\argmin}{argmin}
\DeclareMathOperator{\dist}{dist}
\newcommand{\exercise}[2]{\vspace{1em}\noindent{\bf Exercise #1 (#2)}}
\renewcommand{\proof}{\vspace{0.8em}\noindent{\bf Proof: }}

\begin{document}
\noindent{\footnotesize Game Theory 2014/15, Exercise 4} 
\hfill 
{\footnotesize \input{./currentDate.txt}}
\newline
{\footnotesize \input{../../NAMES.txt}}

\noindent\hrulefill

\emph{
  This is an excerpt that contains only the low-level proof for 4.1 c) and d).
  However, it would be nice if someone could provide a a rigorous proof 
  of the same statement using the duality theorems.
}

\exercise{4.1}{Equalizing strategies}

Let $N=\set{1,2}$ be a set of players, $S_1 := \set{T,M,B}$, 
$S_2 := \set{L, C, R}$ their sets of pure strategies, and 
$u_1,u_2: S_1\times S_2 \to \Real$ their utility functions given by the 
following matrix:
\[
   \mat{cccc}{
       & L       & C       & R \\
    T  & (3, -3) & (-3, 3) & (0, 0) \\
    M  & (2, -2) & (6, -6) & (4, -4) \\
    B  & (2, -2) & (5, -5) & (6, -6).
   }
\]
Let's write $U := (u_1(x, y))_{x\in S_1, y \in S_2}$.
Since this is a zero-sum game, the analogon of the matrix $U$ for player $2$ 
is $-U^T$.

\noindent {\bf a)} We want to find a mixed strategy for player $1$ such
that the expected gain is the same for all pure strategies of player $2$. 
Let's denote this mixed strategy by a $S_1$-valued random variable $X_1$.
That is, we want to show that there exists a constant $c_1\in\Real$ and 
numbers $\P[X_1 = x]$ for $x\in S_1$ such that for all $y\in S_2$ it holds:
\[
  c_1 = \E[u_1(X_1,y)] = \sum_{x\in S_1} \P[X_1 = x] u_1(x, y).
\]
Keeping in mind that it must hold that $\sum_{x}\P[X_1 = x] = 1$, we obtain 
the following linear equation:
\[
  \mat{cccc}{
    u_1(T, L) & u_1(M, L) & u_1(B, L) & -1 \\
    u_1(T, C) & u_1(M, C) & u_1(B, C) & -1 \\
    u_1(T, R) & u_1(M, R) & u_1(B, R) & -1 \\
    1         & 1         & 1         & 0
  }
  \mat{c} {
    \P[X_1 = T] \\
    \P[X_1 = M] \\
    \P[X_1 = B] \\
    c_1
  }
  =
  \mat{c} {
    0 \\ 0 \\ 0 \\ 1
  },
\]
which we write more compactly as
\[
  \mat{cc}{
    U^T & -1_{3\times 1} \\
    1_{1\times 3} & 0
  }
  \mat{c}{p_1 \\ c_1} =
  \mat{c}{0_{3\times 1} \\ 1}
\]
with the obvious meaning of $p_1$. This can be solved with something like
Octave/Matlab as follows:
\begin{lstlisting}
U = [3,-3,0;2,6,4; 2,5,6];
o = [1;1;1];
first = [U',-o;o',0];
second = [-U, -o;o',0];
rhs = [0;0;0;1];
% The result for part a)
disp(first \ rhs);
% The result for part b)
disp(second \ rhs);
\end{lstlisting}
The result is:
\[
  p_1^T = \mat{ccc}{0.4 & 0.6 & 0} \qquad c_1 = 2.4.
\]

\noindent {\bf b)} A completely symmetric derivation for player 2 
leads to the equation
\[
  \mat{cc}{
    -U & -1_{3\times 1} \\
    1_{1\times 3} & 0
  }
  \mat{c}{p_2 \\ c_2} = 
  \mat{c}{0_{3\times 1} \\ 1}
\]
with the solution 
\[
  p_2^T = \mat{ccc}{0.88 & 0.08 & 0.04} \qquad c_2 = -2.4.
\]

\noindent {\bf c) } We want to show that the equalizing strategies 
$X_1$ and $X_2$ (with the properties and probabilities as above) 
are optimal for both players.

First, notice the following property (resulting from what has been called
``multilinearity'' in the lecture). If $Y_2$ is some mixed strategy of the
second player, then it holds:
\[
  \E[u_1(X_1, Y_2)] = \sum_{y\in S_2}\P[Y_2 = y]\E[u_1(X_1,y)]
  = \sum_{y\in S_2}\P[Y_2 = y]c_1 = c_1.
\]
In particular, we obtain just the constant $c_1$ if we minimize over all
possible mixed strategies of player $2$:
\[
  \min_{Y_2} \E[u_1(X_1, Y_2)] = c_1.
\]
A symmetric statement holds for $c_2$. This allows us to make estimations for
Maxmin-values for both players. Here are two inequalities for Maxmin-values of
player $1$:
\begin{align*}
  \max_{Y_1} \min_{Y_2} \E[u_1(Y_1, Y_2)] 
    &\geq \min_{Y_2} \E[u_1(X_1, Y_2)] = \min_{Y_2} c_1 = c_1 \\
  \max_{Y_1}\min_{Y_2}\E[u_1(Y_1, Y_2)] 
    &= \max_{Y_1} \min_{Y_2} \E[-u_2(Y_1, Y_2)] \\
    &\leq \max_{Y_1} \E[-u_2(Y_1, X_2)] \\
    & = \max_{Y_1}(-c_2) \\
    & = -c_2
\end{align*}
In particular, if we have the situation that $c_2 = -c_1$, these two 
inequalities combine into a single equality:
\[
  \max_{Y_1} \min_{Y_2} \E[u_1(Y_1, Y_2)] = c_1.
\]
Exchanging the indices $1$ and $2$ yields a completely symmetric result for
player $2$. 

In parts a) and b) we have found out that for the concrete example we have
$c_1 = -c_2 = 2.4$, so we can conclude that the Maxmin-value for player $1$ is
$2.4$ and the Maxmin-value for player $2$ is $-2.4$, which by definition makes
the equalizing strategies $X_1$ and $X_2$ \emph{optimal}.

\noindent {\bf d)} Now we want to consider a two-player zero-sum game where 
the players $1$ and $2$ have more possible actions. We want to show that if 
both players have equalizing strategies, these strategies are optimal.

Suppose $S_1 = \set{1\dots n}$ and $S_2 = \set{1 \dots m}$ for some natural 
numbers $n,m$. Let $U$ be a $n\times m$ matrix defined as above.
Suppose there are equalizing strategies $X_1$ and $X_2$, that is, 
there exist probability vectors $p_1\in \Real^n$ and $p_2\in\Real^m$ such that
the following holds:
\[
  \mat{cc}{
    U^T & -1_{m\times 1} \\
    1_{1\times n} & 0
  }
  \mat{c}{p_1 \\ c_1} =
  \mat{c}{0_{m\times 1} \\ 1}
\]
\qquad
\[
  \mat{cc}{
    -U & -1_{n\times 1} \\
    1_{1\times m} & 0
  }
  \mat{c}{p_2 \\ c_2} =
  \mat{c}{0_{n\times 1} \\ 1}.
\]
In c) we have shown that it is sufficient to establish $c_2 = -c_1$ in order 
to prove the optimality (the argument did not depend on the dimension of
the matrix $U$ in any way). It holds:
\begin{align*}
c_2 &= \mat{cc}{p_2^T & c_2} \mat{c}{0_{m\times 1} \\ 1} \\
    &= \mat{cc}{p_2^T & c_2} \mat{cc}{
      U^T & -1_{m\times 1} \\ 1_{1\times n} & 0
    } \mat{c}{p_1 \\ c_1} \\
    &= -\rPar{
      \mat{cc}{-U & -1_{n\times 1} \\ 1_{1\times m} & 0}
      \mat{c}{p_2 \\ c_2}
    }^T \mat{c}{p_1 \\ c_1} \\
    &= - \mat{cc}{0_{1\times n} & 1} \mat{c}{p_1 \\ c_1} \\
    &= - c_1
\end{align*}
Hence, the equalizing strategies are optimal. \hfill \qed

\noindent {\bf e)} Finally, we want to consider a counterexample.
Consider the following two-player zero-sum game:
\[
  \mat{ccc}{
       & L        & R \\
     T & (1, -1)  & (1, -1) \\
     B & (-1, 1)  & (-1, 1)
  }
\]
Here, the first player has an equalizing strategy with 
$p_1 = \mat{cc}{0.5 & 0.5}^T$. With this strategy, 
the first player would on average win nothing, 
no matter what the strategy of the second player is.

However, the optimal strategy would obviously be the pure strategy $T$.
Since the decision of the second player can not affect the outcome in any way, 
he can take either $L$ or $R$ as optimal strategy. The value of the game is 
therefore $(1, -1)$, and not $0$, if both play rationally, the first player 
wins $(+1)$ and the second loses $(-1)$.

However, this is not a contradiction to d), because here the second player 
does not have an equalizing strategy: no matter how he weights $L$ and $R$, 
the outcome in the row $T$ is always $(-1)$, 
the outcome in the row $B$ is always $(+1)$, and $(-1) \neq (+1)$.

\end{document}
