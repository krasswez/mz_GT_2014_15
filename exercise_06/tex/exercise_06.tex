\documentclass{scrartcl}

\usepackage{amsmath}	  % required for math in general
\usepackage{amsthm}     % environments for theorems, qed's etc
                        % (loaded after amsmath)
\usepackage{amssymb}	  % doublestroke symbols, other mathematical symbols
\usepackage{dsfont}     % required for double-stroke 1 as characteristic function
\usepackage{array}	    % control of matrices and tables
\usepackage{graphicx}   % images

\usepackage{enumitem}   % more fine-grained control over enumerations
\setdescription{leftmargin=\parindent,labelindent=\parindent}

\usepackage{listings} % code listings
\lstset{basicstyle=\ttfamily\scriptsize}

% \input{diagrams.sty} (no category theory this time)

\usepackage{helvet}   % use (much fresher looking) helvetica for everything
\renewcommand{\familydefault}{\sfdefault}

\usepackage[weather]{ifsym}      % \Lightning symbol
% \usepackage{mathabx}             % \Asterisk causes some conflicts

% forcing the fucking floats to stop fucking floating like a fucking piece of
% shit in an ocean of fucking shit
\renewcommand{\topfraction}{.85}
\renewcommand{\bottomfraction}{.7}
\renewcommand{\textfraction}{.15}
\renewcommand{\floatpagefraction}{.66}
\renewcommand{\dbltopfraction}{.66}

% making all references into hyperlinks
\usepackage[dvipsnames]{xcolor}
\usepackage{hyperref}

\hypersetup{colorlinks=true,linkcolor=MidnightBlue,pdfborderstyle={/W 0}}

\usepackage{anyfontsize}
\usepackage{datetime}

% forall
\let\oldforall\forall
\renewcommand{\forall}{\oldforall\,}

% parentheses
\newcommand{\rPar}[1]{\left(#1\right)} % round parens
\newcommand{\sPar}[1]{\left[#1\right]} % square parens
\newcommand{\cPar}[1]{\left\{#1\right\}} % curved parens 
\newcommand{\aPar}[1]{\left\langle #1 \right\rangle} % angle brackets

% floor and ceiling
\newcommand{\floor}[1]{{\left\lfloor#1\right\rfloor}} % curved parens 
\newcommand{\ceil}[1]{{\left\lceil#1\right\rceil}} % curved parens 

% norms
\newcommand{\abs}[1]{\left\lvert #1\right\rvert}
\newcommand{\norm}[1]{\left\lVert #1\right\rVert}
\newcommand{\scalar}[2]{\left\langle#1,#2\right\rangle}
\newcommand{\cross}{\times}
\DeclareMathOperator{\diam}{diam}
\DeclareMathOperator{\B}{B}

% intervals
\newcommand{\openOpenInterval}[2]{{\left(#1,#2\right)}}
\newcommand{\openClosedInterval}[2]{{\left(#1,#2\right]}}
\newcommand{\closedOpenInterval}[2]{{\left[#1,#2\right)}}
\newcommand{\closedClosedInterval}[2]{{\left[#1,#2\right]}}

% restriction of functions
\newcommand{\restrict}[2]{{\left.#1\right\vert_{#2}}}

% numbers
\newcommand{\Natural}{\mathbb{N}}
\newcommand{\Integer}{\mathbb{Z}}
\newcommand{\Real}{\mathbb{R}}
\newcommand{\Rational}{\mathbb{Q}}
\newcommand{\PositiveReal}{\Real_{>0}}
\newcommand{\NonnegativeReal}{\Real_{\geq0}}
\newcommand{\Complex}{\mathbb{C}}
\renewcommand{\i}{i}
\newcommand{\Quaternion}{\mathbb{H}}
\newcommand{\Boolean}{\mathbb{B}}

% function spaces
\newcommand{\SemiLebesgue}{\mathscr{L}}
\newcommand{\Continuous}{C}
\newcommand{\Lebesgue}{L}
\newcommand{\Sobolev}{H}
\newcommand{\Hilbert}{\mathscr{H}}
\newcommand{\Schwarz}{\mathscr{S}}

% set
\newcommand{\setPredicate}[2]{{\left\{#1\,\left\vert\, #2\right.\right\}}}
\newcommand{\set}[1]{{\left\{#1\right\}}}
\newcommand{\cardinality}[1]{\left\lvert #1 \right\rvert}
\newcommand{\powerset}[1]{\mathfrak{P}\left(#1\right)}
\DeclareMathOperator*{\intersection}{\bigcap}
\DeclareMathOperator*{\union}{\bigcup}
\newcommand{\disjointUnion}{\biguplus}
\renewcommand{\complement}[1]{#1^c}
% \newcommand{\setminus}{\backslash}
\newcommand{\injective}{\hookrightarrow}
\newcommand{\surjective}{\twoheadrightarrow}
%\DeclareMathOperator{\ker}{ker} % already exists... im does not?
\DeclareMathOperator{\im}{im}

% topological operators
\DeclareMathOperator{\Cl}{Cl}
\newcommand{\Closure}[2]{\Cl_{#1}\left(#2\right)}
\DeclareMathOperator{\const}{const}

% span and conv
\DeclareMathOperator*{\conv}{conv}
\DeclareMathOperator*{\linhull}{span}

% matrices
\newcommand{\mat}[2]{\left[\begin{array}{#1}#2\end{array}\right]}
\DeclareMathOperator*{\diag}{diag}

% landau symbols
\newcommand{\LandauO}[1]{\mathcal{O}\left(#1\right)}

% derivatives
\newcommand{\dd}[2]{\frac{\partial #1}{\partial #2}}
\newcommand{\differential}[1]{\boldsymbol{D}_{#1}}

% integrals
\renewcommand{\d}{\quad\mathrm{d}}

% characteristic functions, expected values, variances, covariances
% stochastic stuff
\newcommand{\one}[1]{\mathds{1}_{#1}}
\newcommand{\weakconv}[1]{\overset{#1}{\Longrightarrow}}
\newcommand{\wlim}{\mathop{\mathrm{wlim}}}
\newcommand{\vlim}{\mathop{\mathrm{vlim}}}
\renewcommand{\P}{\mathbb{P}}
\newcommand{\E}{\mathbb{E}}

% lim inf lim sup
% \DeclareMathOperator{\liminf}{lim inf}
% \DeclareMathOperator{\limsup}{lim sup}

% qed etc.
\renewcommand{\qedsymbol}{$\blacksquare$}
\newcommand{\result}{\hfill $\Diamond$}

% lattices
\newcommand{\meet}{\wedge}
\newcommand{\join}{\vee}
\newcommand{\negate}{\neg}

% listings: Scala
\lstdefinelanguage{scala}{
  morekeywords={abstract,case,catch,class,def,%
    do,else,extends,false,final,finally,%
    for,if,implicit,import,match,mixin,%
    new,null,object,override,package,%
    private,protected,requires,return,sealed,%
    super,this,throw,trait,true,try,%
    type,val,var,while,with,yield},
  otherkeywords={=>,<-,<\%,<:,>:,\#,@},
  sensitive=true,
  morecomment=[l]{//},
  morecomment=[n]{/*}{*/},
  morestring=[b]",
  morestring=[b]',
  morestring=[b]"""
}
\lstset{showstringspaces=false}

% making references look a little nices
\let\oldRef\ref
\renewcommand{\ref}[1]{(\oldRef{#1})}

% weird stuff for computer science
\DeclareMathOperator{\arity}{ar}

% cat, category theory
% Bunch of categories
\DeclareMathOperator{\Id}{Id}
\DeclareMathOperator{\Top}{Top}
\DeclareMathOperator{\hTop}{h-Top}
\DeclareMathOperator{\Sets}{Sets}
\DeclareMathOperator{\Rel}{Rel}
\DeclareMathOperator{\FinSets}{FinSets}
\DeclareMathOperator{\Grp}{Grp}
\DeclareMathOperator{\Cat}{Cat}
\DeclareMathOperator{\Grpd}{Grpd}
\newcommand{\cat}[1]{\mathcal{#1}}
\newcommand{\Obj}{\mathrm{Obj}}
\newcommand{\Hom}{\mathrm{Hom}}
\newcommand{\op}{\mathrm{op}}
\newcommand{\nat}{\xrightarrow{\bullet}}
\newcommand{\iso}{\cong}
\newcommand{\dom}{\mathrm{dom}}
\newcommand{\cod}{\mathrm{cod}}
\DeclareMathOperator{\coeq}{Coeq}
\newcommand{\fst}{\mathrm{fst}}
\newcommand{\snd}{\mathrm{snd}}
\DeclareMathOperator{\Aut}{Aut}
\DeclareMathOperator{\End}{End}

% functors frequently used in various contexts
\DeclareMathOperator{\Free}{Free}
\DeclareMathOperator{\Forget}{Forget}

% empty set that is round
\let\emptyset\varnothing

% generated groups
\newcommand{\gen}[1]{\left\langle#1\right\rangle}
\newcommand{\normalSub}{\triangleleft}
\newcommand{\Asterisk}{\mathop{\scalebox{1.5}{\raisebox{-0.2ex}{$\ast$}}}}
\newcommand{\Sym}{\mathrm{Sym}}

% argmax argmin argsup etc.
\DeclareMathOperator{\argsup}{argsup}

% number theoretic operators
\DeclareMathOperator{\lcm}{lcm}

% get rid of the ugly-looking "epsilon"
\renewcommand{\epsilon}{\varepsilon}

% get rid of the empty-looking "angle"
\renewcommand{\angle}{\measuredangle}

\newcommand{\exercise}[2]{\vspace{1em}\noindent{\bf Exercise #1 (#2)}}
\renewcommand{\proof}{\vspace{0.8em}\noindent{\bf Proof: }}

\begin{document}
\noindent{\footnotesize Game Theory 2014/15, Exercise 6} 
\hfill 
{\footnotesize \input{./currentDate.txt}}
\newline
{\footnotesize \input{../../NAMES.txt}}

\noindent\hrulefill

\exercise{6.1}{Properly mixed strategies}
Let $S$ be finite or countable set of pure strategies 
(with at least two elements), 
let $\Sigma := \Delta(S)$ be the mixed strategies.
We call a strategy $\sigma$ \emph{properly mixed} if there does not exist
an $s\in S$ such that $\sigma(s)=1$ and $\sigma(s^\prime)=0$ for all other
$s^\prime\neq s$.

Now let $\sigma\in\Sigma$ be an arbitrary strategy. 
We want to show that there exists a sequence $(\sigma_n)_n$ of properly 
mixed strategies with $\lim_{n\to\infty} \sigma_n = \sigma$.

We show a slightly stronger statement that there exists a sequence of $\sigma_n$
with $\sigma_n(s) > 0$ for \emph{all} $s\in S$.

Suppose that we can find a strategy $\gamma\in\Sigma$ with the property 
$\gamma(s) > 0$ for all $s\in S$.
Define the sequence as follows:
\[
  \sigma_n := \frac{1}{n}\gamma + \rPar{1 - \frac{1}{n}}\sigma.
\]
For all $s\in S$ it holds:
\[
  \sigma_n(s) \geq \frac{1}{n}\gamma(s) > 0,
\]
and in particular, $\sigma_n$ is properly mixed.

This sequence indeed converges to $\sigma$ (e.g. in the $\norm{-}_\infty$-norm,
or actually in any norm $\norm{-}$):
\begin{align*}
  \norm{\sigma_n-\sigma} 
    = \norm{\frac{1}{n}\gamma - \frac{1}{n}\sigma}
    \leq \frac{1}{n}\rPar{\norm{\gamma} + \norm{\sigma}}
    = \frac{\const}{n} \overset{n\to\infty}\longrightarrow 0,
\end{align*}
that is, $\lim_{n\to\infty} \sigma_n = \sigma$.

The remaining question is whether we can obtain a $\gamma$ as above. 
For $S$ finite, we can construct $\gamma$ as follows:
\[
  \gamma(s) := \frac{1}{\abs{S}}.
\]
If $S$ is countably infinite, we can choose some bijection $\psi$ between $S$
and $\Natural$ and define $\gamma$ as follows:
\[
  \gamma(s) := 2^{-\psi(s)},
\]
this is indeed a probability distribution, because:
\begin{align*}
  \sum_{s\in S}\gamma(s) 
    = \sum_{n\in \Natural}\gamma\rPar{\psi^{-1}(s)}
    = \sum_{n\in \Natural}2^{-n} = 1.
\end{align*}
In both cases we obtain a $\gamma$ as required for the construction of the
sequence of the properly mixed strategies. \hfill \qed

\exercise{6.2}{``Perfect'' Nash-Equilibria}
The concept of ``perfect nash-equilibria'' does not seem to occur anywhere 
except the few slides of the lecture. Even the 1k+ pages book on game theory
does not mention it (?). Link to some literature would be highly appreciated.

\exercise{6.3}{Evolutionary Stable Strategies}
Let $N=\set{1, 2}$ be the set of players, $S$ a finite set of strategies (with
at least two elements) and $u_1: \Sigma(S)^2 \to [0, \infty]$ a payoff function. 
The payoff for the second player is assumed to be symmetric: 
$u_2(x, y) = u_1(y, x)$ (we don't need this fact here, but the whole model does
not make any sense otherwise).

\noindent {\bf a)} 
Let $x^\ast\in S$ be a \emph{strict} symmetric Nash-Equilibrium, that is, for
all $x\neq x^\ast$ it holds:
\[
  u_1(x, x^\ast) < u_1(x^\ast, x^\ast).
\]
Then $x^\ast$ is an evolutionary stable strategy, that is: for all 
$x\neq x^\ast$ there exists an $\epsilon_0 > 0$ such that for all 
$\epsilon \in (0, \epsilon_0)$ it holds:
\[
  (1-\epsilon)u_1(x,x^\ast) + \epsilon u_1(x,x) =
  (1-\epsilon)u_1(x^\ast,x^\ast) + \epsilon u_1(x^\ast,x).
\]

\noindent {\bf Proof: } Fix some $x\neq x^\ast$. 
Let $\norm{u_1}_\infty := \max_{s\in \Sigma(S)}\abs{u_1(s)}$ denote the maximum payoff. We choose the $\epsilon_0$
as follows:
\[
  \epsilon_0 := 
    \frac{u_1(x^\ast, x^\ast) - u_1(x, x^\ast)}
    {4\norm{u_1}_\infty},
\]
this choice will becomes obvious after one looks at the
following inequality. For all $\epsilon < \epsilon_0$ it now holds:
\begin{align*}
  &u_1(x, x^\ast) + \epsilon\rPar{
    u_1(x, x) - u_1(x, x^\ast) + u_1(x^\ast, x^\ast) - u_1(x^\ast, x)
  } \\
  &\leq u_1(x, x^\ast) + \epsilon \rPar{4 \norm{u_1}_\infty} \\
  &< u_1(x, x^\ast) + \epsilon_0\rPar{4 \norm{u_1}_\infty} \\
  &= u_1(x, x^\ast) + \rPar{u_1(x^\ast, x^\ast) - u_1(x, x^\ast)} \\
  &= u_1(x^\ast, x^\ast)
\end{align*}
This is exactly equivalent to the definition of ESS (the previous expression 
arises from the next one, again: the original thought process goes in the 
opposite direction of the proof):
\[
  (1-\epsilon)u_1(x, x^\ast) + \epsilon u_1(x, x)
  < 
  (1-\epsilon)u_1(x^\ast, x^\ast) + \epsilon u_1(x^\ast, x).
\]
\hfill \qed

\noindent {\bf Remark: } The sense of this exercise is not to
just say ``that's true, because that's what the book says'',
because the book says (quote): ``verify!'' (Theorem 5.52).

\vspace{1em}

\noindent {\bf b)} Let $\sigma^\ast$ be an ESS and let 
$\sigma^\prime$ be a symmetric Nash-Equilibrium that is 
also a best response to $\sigma^\ast$.
Then $\sigma^\ast = \sigma^\prime$.

\noindent {\bf Proof:} This is a corollary (or rather a 
partial paraphrasing) of the theorem 5.52.
Suppose for the sake of contradiction that 
$\sigma^\prime\neq \sigma^\ast$.
By theorem 5.52 applied to the ESS $\sigma^\ast$ one 
of the following two conditions must hold:
\begin{align*}
  u_1(\sigma^\prime, \sigma^\ast) < 
  u_1(\sigma^\ast, \sigma^\ast) \\
  u_1(\sigma^\prime, \sigma^\ast) = 
  u_1(\sigma^\ast, \sigma^\ast) \wedge
  u_1(\sigma^\prime, \sigma^\prime) <
  u_1(\sigma^\ast, \sigma^\prime).
\end{align*}
If the first condition holds, then $\sigma^\prime$ is not 
a best response to $\sigma^\ast$. 
If the second condition holds, then $\sigma^\prime$ is not
a Nash-Equilibrium (because $\sigma^\ast$ is a possible 
profitable deviation from $\sigma^\prime$).
Both cases yield a contradiction. Therefore it must hold:
$\sigma^\prime = \sigma^\ast$. \hfill \qed

\vspace{1em}

\noindent {\bf c) d)} Omitted. 
\end{document}